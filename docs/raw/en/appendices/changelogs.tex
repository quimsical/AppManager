% SPDX-License-Identifier: GPL-3.0-or-later OR CC-BY-SA-4.0
\chapter{Changelogs}\label{ch:changelogs} %%##$chapter-title>>

\section{v3.1.0 (423)}\label{sec:v3.1.0-(423)}

App Manager v3.1.0 comes with a few new features and a lot of improvements.
Visit \href{intent://settings/about/changelog#Intent;scheme=app-manager;end}{Settings > About > Version/Changelog} for details.

\subsection{Android 13 support}
App Manager now targets Android 13 which means most issues in Android 12 and 13 has been addressed, including SSAID and SAF issues as well as monochrome icons and other theming issues.

\paragraph{Known issue} KeyStore backup/restore not working in Android 12 and later.

\subsection{Introducing freeze/unfreeze}
Enable/disable is replaced with freeze/unfreeze to allow greater control on the behaviours of an app.
It supports suspend, disable and hide functionalities which can be controlled at
\href{intent://settings/rules_prefs/freeze_type#Intent;scheme=app-manager;end}{Settings > Rules > Default freezing method}.
In order to make it easy to freeze or unfreeze an app, shortcuts can also be created from the App Info tab by long clicking on the freeze or unfreeze button.

\subsection{Export app list}
In the Main page, it is now possible to export the list of apps in either XML or Markdown format using batch operations.
In the future, the XML file may also be imported to App Manager.

\subsection{Elliptic Curve Crypography (ECC)}
App Manager now fully supports encrypting backups using ECC in addition to offering AES, RSA and OpenPGP\@.

\subsection{New languages}
Two new languages are added: Korean and Romanian.

\subsection{More list options}
In the main page, more sorting and filtering options are added. Sorting options include sorting the apps by total size,
total data usage, launch count, screen time and last usage time. Filtering options include filtering the apps having
at least one item in the Android KeyStore, filtering apps with URIs granted via SAF, and filtering apps with SSAID\@.

\subsection{Improved handling of mode of operation}
Fixed various issues with ADB pairing, handled incomplete USB debugging. Some rooting methods cannot allow interprocess
communication via Binder. In those cases, ADB mode is used as a fallback method by enabling it automatically if possible.

\subsection{Handling multiple users}
When possible, App Manager will be able to display apps from work profile in no-root mode in addition to allowing basic operations such as launching the app or navigating to the system settings.
For backups, it is now possible to restore backups for other users, but for work profile, some apps may only work properly after re-enabling the work profile.
In the installer page, selecting \textit{All users} will now install the app for all users instead of only the current user.
Finally, in the app info tab, current app can be installed in another profile using the \textit{Install for\dots} option available in the three-dots menu.
This is analogous to the \texttt{pm install-existing} command, thereby, making the installation process a lot faster.

\subsection{Explorer enhancements}
Explorer can now open DEX and JAR files in addition to APK files. Several sorting options as well as folder options are also added as the list options.

\subsection{New tag: WX}
In app info tab, a new tag called WX is added. It is displayed in Android 10 and later if the application targets
Android 9 or earlier. It indicates \href{https://en.wikipedia.org/wiki/W\%5EX}{W\^{}X} violation which allows the app to execute arbitrary executable files either by the modification of executables embedded within the app or by downloading them from the Internet.

\subsection{App ops management}
App ops are now managed automatically to avoid various app ops related crashes in various platforms.
This will also lessen the amount of crashes in an unsupported operating system.

\subsection{Batch uninstallation}
In the Main page, enabled batch uninstallation in no-root mode.

\subsection{Running apps}
Enabled advanced searching. Searching now matches not only app labels but also package names.

\subsection{Interceptor}
Copy the intercepted Intent as am command which can be run from either an ADB shell or a terminal using root with the same effectiveness.

\subsection{Device-specific changes}
\paragraph{Graphene OS} Explicitly handle the Internet permission which is a runtime permission in the OS.
\paragraph{MIUI} Fixed permission denied issues in the installer due to a framework issue introduced in MIUI 12.5.
\paragraph{Motorola} Fixed crashes in the Interceptor page due to a framework issue introduced in Android 11.

\subsection{Others}
\begin{itemize}
    \item Improved Java-Smali conversion by including all the subclasses during conversion
    \item Improved scanning performance in the Scanner page
    \item Improved updating the list of apps in the Main page
    \item Scan all the available paths to detect systemless-ly installed system apps
    \item \texttt{vacuum} SQLite database before opening it for viewing or editing.
\end{itemize}

\section{v3.0.0 (410)}\label{sec:v3.0.0-(410)}

App Manager v3.0.0 comes with a lot of features and improvements.
See \href{intent://settings/changelog#Intent;scheme=app-manager;end}{Settings > Changelog} to see a more detailed changelog.

\subsection{Material 3 and More}
Material 3, somewhat similar to \textit{Material You}, is a significant improvement over Material Design 2 with support for dynamic colours in Android 12 and later.
In addition, many design changes have been made in App Manager without any significant changes in the overall user experience.

\paragraph{Known issue} Switches are still based on Material Design 2 which will be fixed in a future release.

\subsection{Wireless Debugging}
Wireless debugging support has been fully implemented. Head over to \Sref{sec:wireless-debugging} for instructions on how to configure wireless debugging.

\begin{tip}{No-root users}
    Due to auto-detection feature, startup time might be large for no-root users when the mode of operation is set to \textit{auto}.
    Instead, no-root users should select \textit{no-root} instead of \textit{auto}.
\end{tip}

\subsection{Languages}
App Manager is fully translated into Indonesian and Italian languages and can be enabled in settings. Bengali is removed due to lack of translators.

\subsection{Introducing App Explorer}
App Explorer can be used to browse the contents of an application. This includes binary XML files, DEX contents or any other media files.
DEX contents can only be explored in Android Oreo (Android 8) and later. It's also possible to convert an \texttt{.smali} file into \texttt{.java} for a better understanding of the reversed code.
This feature, if not needed, can be disabled in Settings > Enable/disable features.

\subsection{Import Backups from Other Applications}
It is possible to import backups from discontinued or obsolete applications such as Titanium Backup, OAndBackup and Swift Backup (version 3.0 to 3.2).
Go to Setting > Backup/restore to find this option.

\subsection{VirusTotal}
VirusTotal is a widely used tool to scan files and URLs for viruses. In the scanner page and in the running apps page, an option to scan files with VirusTotal has been added.
But the option is hidden by default. To enable the option, it is necessary to obtain an API key from VirusTotal. Go to Settings > VirusTotal API Key for more information.

\begin{warning}{Internet feature}
    This is currently the only feature which require an Internet connection. If you wish to use any Internet feature that might also be added in the future,
    enable \textit{Use the Internet} in Settings > Enable/disable features.
\end{warning}

\subsection{Trigger Profiles from the Automation Software}
As the implementation of routine operations is being delayed, an option to trigger profiles from the external automation software is added.
See \Sref{sec:automating-tasks} for instructions on how to configure profile automation.

\subsection{Improved Application Installer}
Application installer includes several improvements including the ability to downgrade applications in no-root mode, installing multiple applications at once and blocking trackers after installation.
In Android 12 and later, no-root users can update applications without any user interactions.

\subsection{Component Blocking}
It is now possible to configure how App Manager should block a component. Visit Settings > Rules > Default blocking method for more information.
In the components tab, long clicking the block/unblock button opens a context menu which allows per-component blocking in a similar manner.
ADB users can also block the components of a \textit{Test only} app.

\subsection{Advanced Searching}
In some pages, the search bar supports additional searching which includes searching via prefix, suffix or even regular expressions.
In the main page, it is also possible to search for applications using the first letters of each word, e.g.\ \textit{App Manager} can be listed by searching for \textit{am}.

\subsection{Shared Libraries}
Shared libraries tab has received a significant improvements. It can display three types of libraries, such as native, jar and APK files.

\subsection{Make the Best Use of Interceptor}
Activity interceptor can be opened directly from the activities tab by long clicking on the launch button, and similarly, activities can be launched from the activity interceptor page with or without root, for any users.

\begin{tip}{Notice}
    Currently, activities opened via root cannot send the results back to the original applications.
\end{tip}

\subsection{Widget: Screen Time}
Screen time widget is quite similar to Digital Wellbeing's widget by the same name.
It displays the total screen time for the day along with the top three apps from all users.

\subsection{Widget: Clear Cache}
Clear cache widget can be to clear cache from all the applications directly from the home screen.


\section{v2.6.0 (385)}\label{sec:v2.6.0-(385)}

\subsection{Introducing Backups}
Back up/restore feature is now finally out of beta!
Read \hyperref[sec:backup-restore]{the corresponding guide} to understand how it works.

\subsection{Introducing Log Viewer}
\hyperref[subsubsec:log-viewer]{Log viewer} is essentially a front-end for \texttt{logcat}.
It can be used to filter logs by \textit{tag} or \textit{pid} (process ID), or even by custom filters.
Log levels AKA verbosity can also be configured.
You can also save, share and manage logs.

\subsection{Lock App Manager}
\hyperref[subsubsec:screen-lock]{Lock App Manager} with the screen lock configured for your device.

\subsection{Extended Modes for App Ops}
You can set any mode for any app ops that your device supports, either from the
\hyperref[subsec:set-mode-for-app-ops-dots]{1-click ops page} or from the \hyperref[subsubsec:app-ops]{app ops tab}.

\subsection{New Batch Ops: Add to Profile}
You can now easily add selected apps to an existing profile using the batch operations.

\subsection{App Info: Improved}
App info tab now has many options, including the ability to change \hyperref[sec:terminologies]{SSAID}, network policy
(i.e.\ background network usage), battery optimization, etc.
Most of the tags used in this tab are also clickable, and if you click on them, you will be able to look at the current
state or configure them right away.

\subsection{Advanced Sort and Filtering Options in the Main Page}
Sort and filter options are now replaced by \hyperref[subsubsec:main-list-options]{List Options} which is highly configurable,
including the ability to filter using profiles.

\subsection{About This Device}
Interested in knowing about your device in just one page?
Go to the bottom of the \hyperref[subsec:device-info]{settings page}.

\subsection{Enable/disable Features}
Not interested in all the features that AM offers?
You can disable some features in \hyperref[subsubsec:enable/disable-features]{settings}.

\subsection{New Languages}
AM now has more than 19 languages!
New languages include Farsi, Japanese and Traditional Chinese.

\subsection{Signing the APK Files}
You can now import external signing keys in AM! For security, App Manager has its own encrypted KeyStore which can also be \hyperref[subsubsec:import/export-keystore]{imported or exported}.

\subsection{New Extension: UnAPKM}
Since APKMirror has removed encryption from their APKM files, it's no longer necessary to decrypt them.
As a result, the option to decrypt APKM files has been removed.
Instead, this option is now provided by the UnAPKM extension which you can grab from \href{https://f-droid.org/packages/io.github.muntashirakon.unapkm/}{F-Droid}.
So, if you have an encrypted APKM file and have this extension installed, you can open the file directly in AM\@.


\section{v2.5.20 (375)}\label{sec:v2.5.20-(375)}

\subsection{Introducing Profiles}\label{subsec:introducing-profiles}
\hyperref[sec:profile-page]{Profiles} finally closes the
\href{https://github.com/MuntashirAkon/AppManager/issues/72}{related issue}. Profiles can be used to execute certain
tasks repeatedly without doing everything manually. A profile can be applied (or invoked) either from the
\hyperref[sec:profiles-page]{Profiles page} or from the home screen by creating shortcuts. There are also some presets
which consist of debloating profiles taken from
\href{https://gitlab.com/W1nst0n/universal-android-debloater}{Universal Android Debloater}.

\paragraph{Known limitations}
\begin{itemize}
    \item Exporting rules and applying permissions are not currently working.
    \item Profiles are applied for all users.
\end{itemize}

\subsection{The Interceptor}\label{subsec:the-interceptor}
\href{https://github.com/MuntashirAkon/intent-intercept}{Intent Intercept} works as a man-in-the-middle between source
and destination, that is, when you open a file or URL with another app, you can see what is being shared by opening it
with Interceptor first. You can also add or modify the intents before sending them to the destination. Additionally,
you can double-click on any exportable activities in the Activities tab in the App Details page to open them in the
Interceptor to add more configurations.

\paragraph{Known limitation} Editing extras is not currently possible.

\subsection{UnAPKM: DeDRM the APKM files}\label{subsec:unapkm:-dedrm-the-apkm-files}
When I released a small tool called \href{https://f-droid.org/en/packages/io.github.muntashirakon.unapkm}{UnAPKM},
I promised that similar feature will be available in App Manager. I am proud to announce that you can open APKM files
directly in the App Info page or convert them to APKS or install them directly.

\subsection{Multiple user}\label{subsec:multiple-user}
App manager now supports multiple users! For now, this requires root or ADB. But no-root support is also being
considered. If you have multiple users enabled and click on an app installed in multiple profiles, an alert prompt will
be displayed where you can select the user.

\subsection{Vive la France!}
Thanks to the contributors, we have one more addition to the language club: French. You can add more languages or
improve existing translations at \href{https://hosted.weblate.org/engage/app-manager}{Weblate}.

\subsection{Report crashes}
If App Manager crashes, you can now easily report the crash from the notifications which opens the share options.
Crashes are not reported by App Manager, it only redirects you to your favourite Email client.

\subsection{Android 11}
Added support for Android 11. Not everything may work as expected though.

\subsection{App Installer Improvements}

\subsubsection{Set installation locations}
In settings page, you can set install locations such as auto (default), internal only and prefer external.

\subsubsection{Set APK installer}
In settings page, you can also set default APK installer (root/ADB only) instead of App Manager.

\subsubsection{Multiple users}
In settings page, you can allow App Manager to display multiple users during APK installation.

\subsubsection{Signing APK files}
In settings page, you can choose to sign APK files before installing them. You can also select which signature scheme
to use in the \textit{APK signing} option in settings.

\paragraph{Known limitation} Currently, only a generic key is used to sign APK files


\section{v2.5.17 (368)}

\subsection{App Installer}
As promised, it is now possible to select splits. AM also provides recommendations based on device configurations.
If the app is already installed, recommendations are provided based on the installed app. It is also possible to
downgrade to a lower version without data loss if the device has root or ADB. But it should be noted that not all app
can be downgraded. Installer is also improved to speed up the installation process, especially, for root users.
If the app has already been installed and the new (x)apk(s) is newer or older or the same version with a different
signature, AM will display a list of changes similar to \textbf{What's New} before prompting the user to install
the app. This is useful if the app has introduced tracker components, new permissions, etc.

\paragraph{Known Limitations}
\begin{itemize}
    \item Large app can take a long time to fetch app info, and therefore, it may take a long time display the installation prompt.
    \item If the apk is not located in the internal storage, the app has to be cached first which might also take
    a long time depending on the size of the apk.
\end{itemize}

\subsection{Scanner: Replacement for Exodus Page}
Exodus page is now replaced with scanner page. \hyperref[sec:scanner-page]{Scanner page} contains not only a list of
trackers but also a list of used libraries. This is just a start. In the future, this page will contain more in depth
analysis of the app.

\subsection{Introducing System Config}
System Config lists various system configurations and whitelists/blacklists included in Android by either OEM/vendor,
AOSP or even some Magisk modules. Root users can access this option from the overflow menu in the main page.
There isn't any official documentation for these options therefore it's difficult to write a complete documentation
for this page. I will gradually add documentations using my own knowledge. However, some functions should be
understandable by their name.

\subsection{More Languages}
Thanks to the contributors, AM now has more than 12 languages. New languages include Bengali, Hindi, Norwegian, Polish,
Russian, Simplified Chinese, Turkish and Ukrainian.

\subsection{App Info Tab}
More tags are added in the \hyperref[subsec:app-info-tab]{app info tab} such as \textbf{KeyStore} (apps with KeyStore
items), \textbf{Systemless app} (apps installed via Magisk), \textbf{Running} (apps that are running). For external apk,
two more options are added namely \textbf{Reinstall} and \textbf{Downgrade}. Now it is possible to share an apk via
Bluetooth. For system apps, it is possible to uninstall updates for root/ADB users. But like the similar option in the
system settings, this operation will clear all app data. As stated above, exodus has been replaced with scanner.

\subsection{Navigation Improvements}
It's now relatively easy to navigate to various UI components using keyboard. You can use up/down button to navigate
between list items and tab button to navigate to UI components inside an item.

\subsection{Running Apps Page}
It is now possible to sort and filter processes in this tab. Also, the three big buttons are replaced with an
easy-to-use three dot menu. Previously the memory usage was wrong which is fixed in this version.

\subsection{Built-in Toybox}
Toybox (an alternative to busybox) is bundled with AM. Although Android has this utility built-in from API 23, toybox
is bundled in order to prevent buggy implementations and to support API < 23.

\subsection{Component Blocker Improvements}
Component blocker seemed to be problematic in the previous version, especially when global component blocking is enabled.
The issues are mostly fixed now.

\begin{warning}{Caution}
    The component blocking mechanism is no longer compatible with v2.5.6 due to various security issues. If you have
    this version, upgrade to v2.5.13 or earlier versions first. After that, enable
    \hyperref[subsubsec:instant-component-blocking]{global component blocking} and disable it again.
\end{warning}

\subsection{Improvements in the App Details Page}
Value of various app ops depend on their parent app ops. Therefore, when you allow/deny an app op, the parent of the app
op gets modified. This fixes the issues some users have been complaining regarding some app ops that couldn't be changed.

If an app has the target API 23 or less, its permissions cannot be modified using the \texttt{pm grant \ldots} command.
Therefore, for such apps, option to toggle permission has been disabled.

The signature tab is improved to support localization. It also displays multiple checksums for a signature.

\subsection{App Manifest}
Manifest no longer crashes if the size of the manifest is too long. Generated manifest are now more accurate than before.


\section{v2.5.13 (348)}

\subsection{Bundled App (Split APK)}
Bundled app formats such as \textbf{apks} and \textbf{xapk} are now supported. You can install these apps using
the regular installation buttons. For root and adb users, apps are installed using shell, and for non-root users,
the platform default method is used.

\paragraph{Known Limitations}
\begin{itemize}
    \item Currently \textit{all} splits apks are installed. But this behaviour is going to change in the next release.
    If you only need a few splits instead of all, extract the \textbf{APKS} or \textbf{XAPK} file, and then, create a new zip
    file with your desired split apks and replace the \textbf{ZIP} extension with \textbf{APKS}. Now, open it with AM\@.
    \item There is no progress dialog to display the installation progress.
\end{itemize}

\subsection{Direct Install Support}
You can now install \textbf{APK}, \textbf{APKS} or \textbf{XAPK} directly from your favourite browser or file manager.
For apps that need updates, a \textbf{What's New} dialog is displayed showing the changes in the new version.

\paragraph{Known Limitations}
\begin{itemize}
    \item Downgrade is not yet possible.
    \item There is no progress dialog to display the installation progress. If you cannot interact with the current page,
    wait until the installation is finished.
\end{itemize}

\subsection{Remove All Blocking Rules}
In the Settings page, a new option is added which can be used to remove all blocking rules configured within App Manager.

\subsection{App Ops}
\begin{itemize}
    \item App Ops are now generated using a technique similar to AppOpsX. This should decrease the loading time
    significantly in the App Ops tab.
    \item In the App Ops tab, a menu item is added which can be used to list only active app ops without including the
    default app ops. The preference is saved in the shared preferences.
\end{itemize}

\paragraph{Known Limitation} Often the App Ops tab may not be responsive. If that's the case, restart App Manager.

\subsection{Enhanced ADB Support}
ADB shell commands are now executed using a technique similar to AppOpsX (This is the \textit{free} alternative of AppOps by Rikka.).
This should dramatically increase the execution time.

\paragraph{Known Limitation} AM can often crash or become not responsive. If that's the case, restart App Manager.

\subsection{Filtering in Main Page}
Add an option to filter apps that has at least one activity.

\subsection{Apk Backup/Sharing}
Apk files are now saved as \texttt{app name\_version.extension} instead of \texttt{package.name.extension}.

\subsection{Batch Ops}
\begin{itemize}
    \item Added a foreground service to run batch operations. The result of the operation is displayed in a notification.
    If an operation has failed for some packages, clicking on the notification will open a dialog box listing the failed
    packages. There is also a \textbf{Try Again} button on the bottom which can be used to perform the operation again for the failed packages.
    \item Replaced Linux \textit{kill} with \textbf{force-stop}.
\end{itemize}

\subsection{Translations}
Added German and Portuguese (Brazilian) translations.

\paragraph{Known Limitation} Not all translations are verified yet.

\subsection{App Data Backup}
Install app only for the current user at the time of restoring backups. Support for split apks is also added.

\textit{Data backup feature is now considered unstable. If you encounter any problem, please report to me without hesitation.}
